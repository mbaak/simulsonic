\documentclass{article}

% if you need to pass options to natbib, use, e.g.:
%     \PassOptionsToPackage{numbers, compress}{natbib}
% before loading neurips_2020

% ready for submission
% \usepackage{neurips_2020}

% to compile a preprint version, e.g., for submission to arXiv, add add the
% [preprint] option:
\usepackage[preprint]{neurips_2020}

% to compile a camera-ready version, add the [final] option, e.g.:
%     \usepackage[final]{neurips_2020}

% to avoid loading the natbib package, add option nonatbib:
 %\usepackage[nonatbib]{neurips_2020}

\usepackage[utf8]{inputenc} % allow utf-8 input
\usepackage[T1]{fontenc}    % use 8-bit T1 fonts
\usepackage{hyperref}       % hyperlinks
\usepackage{url}            % simple URL typesetting
\usepackage{booktabs}       % professional-quality tables
\usepackage{amsfonts}       % blackboard math symbols
\usepackage{nicefrac}       % compact symbols for 1/2, etc.
\usepackage{microtype}      % microtypography

\usepackage{natbib}
\bibliographystyle{unsrtnat}


\title{A fast and high fidelity non-Deep Learning synthetic data generator for tabular data}

% The \author macro works with any number of authors. There are two commands
% used to separate the names and addresses of multiple authors: \And and \AND.
%
% Using \And between authors leaves it to LaTeX to determine where to break the
% lines. Using \AND forces a line break at that point. So, if LaTeX puts 3 of 4
% authors names on the first line, and the last on the second line, try using
% \AND instead of \And before the third author name.

\author{%
  Max Baak \\
  ING Bank\\
  Wholesale Banking Advanced Analytics (WBAA)\\
  Bijlmerdreef 106\\
1102 CT Amsterdam\\ 
Netherlands\\
  \texttt{Max.Baak@ing.com} \\
  % examples of more authors
  % \And
  % Coauthor \\
  % Affiliation \\
  % Address \\
  % \texttt{email} \\
  % \AND
  % Coauthor \\
  % Affiliation \\
  % Address \\
  % \texttt{email} \\
  % \And
  % Coauthor \\
  % Affiliation \\
  % Address \\
  % \texttt{email} \\
  % \And
  % Coauthor \\
  % Affiliation \\
  % Address \\
  % \texttt{email} \\
}

\begin{document}

\maketitle

\begin{abstract}
  Current state-of-the-art approaches towards generating synthetic versions of real, tabular, approximately i.i.d. data by and large make use of the immense progress and representational power of Generative Adversarial Networks and Variational AutoEncoders. We present an approach to the problem which generates results comparable or superior to the existing approaches, with a computational cost an order of magnitude (\&\&\& CHECK THIS FROM SIMON'S RESULTS) lower.
\end{abstract}

\section{Introduction}

One of the current bottlenecks to further progress in deployment of machine learning applications are the issues surrounding protecting the privacy of sensitive data, and also respecting data license agreements which may prohibit the duplication of a licensed dataset. These barriers manifest as delays -even within a given institution- in providing access to data to the Data Scientists who require it, or even being able to create a training set when the data licensing requirements constrain use to a single copy of the data, such that this data can be used in production only, and surrogate, synthetic data is required to train in development, testing, and acceptance environments of the DTAP testing and deployment approach.\footnote{The even more ambitious use case of using synthetic data generators to mitigate overfitting by avoiding data reuse and training on endless synthetically produced training sets remains elusive until synthetic data generation tools are deemed reliable enough to warrant the leap of faith of treating them as being truly almost as good as the true underlying data generation mechanism.} A major stumbling block in fostering further collaboration between industry and academia, or indeed with the wider Data Science community is often the inability to share data due to privacy issues, which is something reliable synthetic data generation could alleviate. 

In order to safely enable the use of realistic but entirely synthetic datasets in these scenarios, reliable tools to model and sample from estimated forms of the empirical joint distributions of the data are required. It is towards this goal that the present work makes progress, with focus on tabular, numeric, approximately independent and identically distributed data.

In the subsequent sections we detail a novel method to learn and generate synthetic data from an existing original dataset, where use of Deep Learning or other high complexity models is entirely optional. The computational cost of fitting is therefore significantly reduced, but without hampering the fidelity of the generated samples, as judged by multiple performance metrics. In section ~\ref{prior-work-section} an overview of prior work related to the present work is overviewed, in section ~\ref{method-section} we detail the method used, and section ~\ref{results-section} presents the results obtained. Potential paths for further work are outlined in section ~\ref{outlook-section}.

POSSIBLY A SENTENCE ON THE USE CASE WITHIN HEP WHICH FIRST INITIATED THIS WORK, MAX? NOT REQUIRED OR STRICTLY RELEVANT, BUT MIGHT BE OF INTEREST TO THE READER IN UNDERSTANDING THE DISTINCT APPROACH TAKEN.


\subsection{Prior work}
\label{prior-work-section}

Efforts to produce realistic synthetic datasets span a wide array of methods, from re-sampling techniques and perturbations applied to the original dataset, to datasets generated from manually curated rules. For a review of some of the existing non-Deep Learning approaches see \cite{surendra2017review}.

In recent years however, encouraged perhaps by the astounding success of GANs and VAEs in successfully generating sythetic images which are fully sampled from their learned weights and yet virtually indistinguishable from real images (for GANs see ~\cite{karras2020analyzing} and for VAEs see ~\cite{razavi2019generating}), approaches of this type have been created for tabular data, with a benchmark emerging for the evaluation and comparison of these methods ~\cite{xu2019modeling}.

In particular, focused on potential applications within finance, of the type which motivated the present work, the NeurIPS 2019 conference included an overview of the field and some of its challenges ~\cite{assefa2020generating}.


\section{Method}
\label{method-section}

The method broadly consists of two stages: learning and removal of linear and distributional characteristics of the data, followed by learning of the residual non-uniform associations in the data. Once these two stages have been fit, the process can then be reversed, generating non-uniform associations as learnt from the data, and regenerating the linear and distributional characteristics of the data, incorporating the necessary Jacobian factors as needed, and thereby effectively sampling from the learned joint distribution.

The code for our method is available at: \url{https://github.com/mbaak/synthsonic}.


\subsection{Quantile transformation to Normality}

Describe Quantile transformation to Normality, motivated by the fact that PCA effectively assumes Normal noise ~\cite{tipping1999probabilistic}

\subsection{Rotation of linear features}

PCA used to rotate-away and learn linear features

\subsection{Quantile transformation to uniformity}

Any remaining distributional shape is fitted and removed

\subsection{Fitting of residuals}

A classifier is trained to distinguish between residuals remaining after the procedure above, and uniform datapoints.
This classifier is ensured to be well-calibrated, describe how

\subsection{Closed form fitted probability density function}

Derivation of Jacobian required and final form of pdf

\subsection{Treatment of categoricals}

Describe how those were dealt with here



\section{Results}
\label{results-section}

We evaluated the method with respect to multiple criteria: with regards to a likelihood criteria as set out in the SDGym benchmark and leaderboard in ~\cite{xu2019modeling}, as well as ensuring marginals of the learnt distributions matched the original datasets, multiple modes -if existent- were preserved, and lastly also in an adverserial way, judged by a discriminating model's ability to distinguish between the synthetic and the original dataset (TO DO).

Lastly, in order to mitigate the risk of having overfit to the SDGym datasets, we also evaluated its performance on synthetic datasets which we manually created, with well known design features, to test the method's ability to recover these and therefore generalise beyond the datasets it was initially applied to.

\subsection{The SDGym benchmark}

We beat all the approaches in ~\cite{xu2019modeling} (VERIFY THIS)

\subsection{Marginals}

Plots of marginals showing overlap between real and fake datasets  
Show at least one multi-modal fit as well

\subsection{Adversarial validation}


\section{Outlook}
\label{outlook-section}

Improve treatment of categoricals?  
Further improvements on best choice of classifier  
Anything else  

SECTIONS BELOW STILL TO BE FILLED OUT, BUT ARE PRO FORMA


\section*{Broader Impact}

Authors are required to include a statement of the broader impact of their work, including its ethical aspects and future societal consequences. 
Authors should discuss both positive and negative outcomes, if any. For instance, authors should discuss a) 
who may benefit from this research, b) who may be put at disadvantage from this research, c) what are the consequences of failure of the system, and d) whether the task/method leverages
biases in the data. If authors believe this is not applicable to them, authors can simply state this.

Use unnumbered first level headings for this section, which should go at the end of the paper. {\bf Note that this section does not count towards the eight pages of content that are allowed.}

\begin{ack}
Use unnumbered first level headings for the acknowledgments. All acknowledgments
go at the end of the paper before the list of references. Moreover, you are required to declare 
funding (financial activities supporting the submitted work) and competing interests (related financial activities outside the submitted work). 
More information about this disclosure can be found at: \url{https://neurips.cc/Conferences/2020/PaperInformation/FundingDisclosure}.


Do {\bf not} include this section in the anonymized submission, only in the final paper. You can use the \texttt{ack} environment provided in the style file to autmoatically hide this section in the anonymized submission.
\end{ack}

\section*{References}

References follow the acknowledgments. Use unnumbered first-level heading for
the references. Any choice of citation style is acceptable as long as you are
consistent. It is permissible to reduce the font size to \verb+small+ (9 point)
when listing the references.
{\bf Note that the Reference section does not count towards the eight pages of content that are allowed.}
\medskip

\bibliography{paper}


\end{document}
